\documentclass[twocolumn,10pt,final]{asme2ej}



%% End of ecrc-specific commands
%%%%%%%%%%%%%%%%%%%%%%%%%%%%%%%%%%%%%%%%%%%%%%%%%%%%%%%%%%%%%%%%%%%%%%%%%%

%% The amssymb package provides various useful mathematical symbols
\usepackage{amssymb}
%% The amsthm package provides extended theorem environments
%\usepackage{amsthm}
\usepackage{amsmath}

% if you have landscape tables
\usepackage[figuresright]{rotating}

\begin{document}

\begin{center}

\textbf{\large{Rotating Scatter Mask Optimization for Gamma Source Direction Identification}}

\noindent
\begin{tabular}{l}
\rule{3.2in}{0.02in}
\end{tabular}

{Darren E. Holland$^a$, James E. Bevins$^b$, Larry W. Burggraf$^b$, Buckley E. O'Day$^b$}
\vspace{0.1in}

{[a] Department of Mechanical Engineering\\
	Cedarville University, Cedarville, OH 45502\\}
{[b] Department of Engineering Physics\\
	Air Force Institute of Technology,	WPAFB, OH 45434\\}

\end{center}

\vspace{-0.4 cm}
\section*{Abstract}
\textbf{
Rotating scattering masks (RSMs) have shown promise as an inexpensive method for identifying a gamma source's direction.  
However, further examination of the current RSM design shows that changing the geometry may improve the identification by 
reducing or eliminating degenerate solutions and lowering the required count times.  
Three approaches are introduced to generate alternative mask geometries.
The eigenvector method uses a spring-mass system to create a geometry basis.  The binary approach uses ones and zeros to represent the geometry.  
Finally, a Hadamard matrix is modified to examine a decoupled geometric solution.   
Four criteria, the max and average modal assurance criteria, efficiency, and sensitivity, are proposed for evaluating the new RSM designs generated from each of these methodologies.
An analysis of the resulting detector response matrices demonstrates that the binary and eigenvector methodologies produced 
superior identification characteristics as compared to existing RSM designs.}

%% main text
\vspace{-0.3 cm}
\section{Introduction}
\label{intro}
Identifying a gamma source's direction is important in a variety of applications such as providing security, treaty compliance verification, and locating orphan sources.  
A novel method, similar in concept to the coded aperture system, was introduced that eliminates many of the limitations of alternative approaches.
This system utilizes a mask placed over a single position-insensitive detector \cite{FitzGerald2015}.  
The system records energy spectra as a function of the geometrically varying mask, which is accomplished through a set, constant mask rotation. 
The obtained position dependent spectra, referred to as detector response curves (DRCs), depend on the source position and can be used to identify the source direction.  FitzGerald's mask geometry results in some DRCs that are nearly identical, which can lead to mis-identification of the source direction.  
This work seeks to reduce the DRCs' linear dependence by optimizing the mask's geometry.

\vspace{-0.3 cm}
\section{RSM Design} \label{sec:rsm-design}
This work uses MCNP to simulate the experimental DRCs needed for evaluating each RSM design's performance.  
Instead of using only the full energy peak (FEP), the DRC is formed by summing all counts above 200keV to increase the source direction identification's efficiency.  
The 200keV limit was chosen as Logan et al. noted discrepancies for counts below this value due to scatter in environmental elements not considered in the model \cite{Logan2017}.

The RSM design is to be optimized for a $^{137}$Cs point source located 36 inches from the center of the detector, mimicking Logan et. al's setup \cite{Logan2017}.  
To simulate the relative source rotation in MCNP, the mask is stationary, while the source is rotated in spherical coordinates every $11.25^\circ$ increment for $\theta$ from 0 to $348.75^\circ$ and each $5.625^\circ$ increment for $\phi$ from 5.625$^\circ$ to $168.75^\circ$.  
To increase the solution convergence rate, particles were emitted within a $27.26^\circ$ half angle cone extending from the source to the detector's center.  

\vspace{-0.2 cm}
\subsection{Design Assumptions and Limitations}
For this application, it is desirable that the curve generated at each initial source position be orthogonal - a.k.a. unique - to the others.  
Let this curve be denoted $\mathbf{DRC}_{i,j},$ where $i=0,1,...,n$ is the initial $\theta$ and $j=0,1,...,m$ is the initial $\phi$ index relative to a reference location on the mask.  
Since the mask rotates, the $i^{th}$ DRC will be identical to $\mathbf{DRC}_{g,j}$ shifted by $g-i$ indices.  
A negative number corresponds to a shift to the left and a positive number a shift to the right.  
This property greatly impacts the mask design as any periodic vector with respect to $\theta$ will then result in duplicate $i$ and $g$ DRCs.  
The duplication due to periodicity would fail to meet the design's uniqueness requirements.

\vspace{-0.2 cm}
\subsection{Modal Assurance Criterion} \label{sec:MAC}
The modal assurance criterion (MAC) is a normalized number that indicates the similarity between two vectors \cite{Allemang03}.
A MAC value of zero indicates the two vectors are orthogonal, while a value of one indicates that they are identical.  

Thus, assuming the measured response can be represented by the simulated spectrum, it is possible to analyze the uniqueness of the design by comparing each DRC with every other possible DRC to find the worst performance.  Considering all vector shifts, the maximum MAC number, which corresponds to the most similar pair of DRCs, is given in Eqn.~\ref{eq:maxmac}.
\begin{equation}
M=max_{g,h,i,j}\left(\frac{\left(\mathbf{u}_{g,h}^T\mathbf{v}_{i,j}\right)^2}{\left(\mathbf{u}_{g,h}^T\mathbf{u}_{g,h}\right)\left(\mathbf{v}_{i,j}^T\mathbf{v}_{i,j}\right)}\right),
\label{eq:maxmac}
\end{equation}
\noindent where $\mathbf{u}_{g,h}$ and $\mathbf{v}_{i,j}$ are the DRCs for the respective initial positions $\left(\theta=g\Delta\theta,\phi=h\Delta\phi\right)$ and
$\left(\theta=i\Delta\theta,\phi=j\Delta\phi\right)$, and $g\neq i$ and $h\neq j$.

\vspace{-0.2 cm}
\subsection{Design Methodologies} \label{design-methods}
It is impossible to create a full set of linearly independent DRCs.  
Therefore, the optimized design objective is to create a design with the least amount of linear dependence.  
The three following methods are tailored to create designs for identifying both $\theta$ and $\phi$ with a low linear dependence.  
The eigenvector approach solves a mass normalized eigenvalue problem.  
The binary method uses patterns of ones and zeros to represent the geometry.
Lastly, the Hadamard approach used for rotating encoding masks is modified and applied.

\vspace{-0.2 cm}
\subsection{Design Evaluation}
\label{sec:Eval}
To assess the design optimality, four criteria relevant to the performance of the RSM are proposed.  The first, the maximum MAC value, was discussed in Section~\ref{sec:MAC}.  
In addition, the average MAC number, $A$, provides information about the mean linear dependence.  
The third criteria measures the RSM's average efficiency ($\epsilon$) for a mask cell. 
A high-efficiency design produces more accurate results in less measurement time, an important factor for the intended RSM applications.  
The final evaluation criterion focuses on the design's sensitivity (S).  
The ratio of the maximum to minimum response provides information on the relative amount of measurement time required and the  sensitivity to random measurement noise.
The FitzGerald's RSM was used as a baseline to establish the design improvement using these proposed criteria.  

\vspace{-0.3 cm}
\section{Results and Analysis}
The eigenvector approach, shown in Fig.~\ref{fig:EVGeo}, used 32 stiffness values ranging from 0.1 to -0.2999 in increments of -0.0129 with stiffness coupling to the nearest two neighbors on both sides of the mass.
A linear dependence scalar was determined to be 0.999 from a design geometry minimization.

\vspace{-0.2 cm}
\begin{figure}[ht!]
\centering
\includegraphics[width={1.85in}]{../figs/EVGeo.pdf}
\caption{RSM geometry created by the eigenvector approach.}
\label{fig:EVGeo}
\end{figure}
\vspace{-0.3 cm}

The binary pattern was constructed to have a fin that spirals around the mask as shown in Fig~\ref{fig:BiGeo}.  
As it is not possible to have the fin completely cover the mask geometry, four other vectors were added to create the necessary basis. 

\vspace{-0.2 cm}
\begin{figure}[ht!]
\includegraphics[width={1.85in}]{../figs/BiGeo.pdf}
\centering
\caption{RSM geometry created by the binary approach.}
\label{fig:BiGeo}
\end{figure}
\vspace{-0.3 cm}

The Hadamard approach was used to create a design, shown in Fig~\ref{fig:HadGeo}, with $m$ linearly independent geometric basis vectors for $\phi$ identification.
The reduced basis set was made possible through decoupling $\theta$ and $\phi$ with the introduction of additional material to create a low measurement ``dead" zone at a consistent $\theta$ position.  
The evaluation results for the original RSM as well as the binary, eigenvector (EV), and Hadamard approaches are summarized in Table~\ref{table:results}.

\vspace{-0.2 cm}
\begin{figure}[ht!]
\centering
\includegraphics[width={1.85in}]{../figs/HadGeo.pdf}
\caption{RSM geometry created by the Hadamard approach.}
\label{fig:HadGeo}
\end{figure}
\vspace{-0.3 cm}
  
\vspace{-0.6 cm}
\begin{table}[ht]
\caption{Evaluation criteria comparisons from the original design and the three proposed designs from this work.} % title of Table
\centering % used for centering table
\begin{tabular}{|c|c|c|c|c|} % centered columns (4 columns)
\hline
Criteria & Fitzgerald & EV & Binary & Hadamard\\
\hline
M & 1.00 & 0.808 & 0.963 & 0.935\\
\hline
A & 0.210 & 0.0663 & 0.125 & 0.423 \\
\hline
$\epsilon \left(\times 10^{-4}\right)$ & 3.75 & 2.96 & 3.86 & 3.70\\
\hline
S & 1.00 & 1.07 & 1.16 & 1.07 \\
\hline
\end{tabular}
\label{table:results} % is used to refer this table in the text
\end{table}
\vspace{-0.3 cm}

All of the proposed methods have lower $M$ values with the most desirable corresponding to the eigenvector approach.  
The Hadamard method produces vectors which on average share~42.3\% of their information, which is more than the original design.  
The rapid change between ones and zeros coupled with the spatial resolution limits make the Hadamard method non-ideal for this problem.  
The lowest average MAC number corresponds to the eigenvector approach with only~6.63\% similarity.  
These results indicate that the method that produces the most unique
DRCs is the eigenvector approach.  
However, the trade-off is that the eigenvector method produced a RSM with a lower average normalized number of counts per cell.  
The binary method has the most average normalized number of counts per cell, but also the highest sensitivity.  
This high sensitivity may indicate that certain initial positions are more susceptible to measurement noise.

%% References
%%
%% Following citation commands can be used in the body text:
%% Usage of \cite is as follows:
%%   \cite{key}         ==>>  [#]
%%   \cite[chap. 2]{key} ==>> [#, chap. 2]
%%

%% References with BibTeX database:

\vspace{-0.3 cm}
\bibliographystyle{elsarticle-num}
\bibliography{RSM}

%% Authors are advised to use a BibTeX database file for their reference list.
%% The provided style file elsarticle-num.bst formats references in the required Procedia style

%% For references without a BibTeX database:

% \begin{thebibliography}{00}

%% \bibitem must have the following form:
%%   \bibitem{key}...
%%

% \bibitem{}

% \end{thebibliography}

\end{document}

%%
%% End of file `ecrc-template.tex'. 